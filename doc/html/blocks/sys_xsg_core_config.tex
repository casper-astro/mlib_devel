\documentclass{article}
\oddsidemargin  0.0in
\evensidemargin 0.0in
\textwidth      6.5in
\usepackage{tabularx}
\usepackage{html}
\title{\textbf{CASPER Library} \\Reference Manual}
\newcommand{\Block}[6]{\section {#1 \emph{(#2)}} \label{#3} \textbf{Block Author}: #4 \\ \textbf{Document Author}: #5 \subsection*{Summary}#6}

\newenvironment{PortTable}{\subsection*{Ports}
\tabularx{6.5in}{|l|l|l|X|} \hline  \textbf{Port} & \textbf{Dir.} & \textbf{Data Type} & \textbf{Description} \\ \hline}{\endtabularx}

\newcommand{\Port}[4]{\emph{#1} & \lowercase{#2} & #3 & #4\\  \hline}

\newcommand{\BlockDesc}[1]{\subsection*{Description}#1}

\newenvironment{ParameterTable}{\subsection*{Mask Parameters}
\tabularx{6.5in}{|l|l|X|} \hline  \textbf{Parameter} & \textbf{Variable} & \textbf{Description} \\ \hline}{\endtabularx}

\newcommand{\Parameter}[3]{#1 & \emph{#2} & #3 \\ \hline}

\begin{htmlonly}
\newcommand{\tabularx}[3]{\begin{tabularx}{#1}{#2}{#3}}
\newcommand{\endtabularx}{\end{tabularx}}
\end{htmlonly}

\date{Last Updated \today}
\begin{document}
\maketitle

%\chapter{System Blocks}
%%%%Change Chapter%%%%%%%%
%\chapter{Signal Processing Blocks}

%\input{test.tex}
%\chapter{Communication Blocks}
%\end{document} 
% XSG Core Config block
% Last modified: 2007/06/23

\Block{XSG Core Config}{XSG core config}{xsgcoreconfig}
{Pierre-Yves Droz}{Henry Chen}
{The XSG Core Config block is used to configure the System Generator design
for the {\em bee\_xps} toolflow. Settings here are used to configure the
Xilinx System Generator block parameters automatically, and control toolflow
script  execution. It needs to be at the top level of all designs being compiled
with the {\em bee\_xps} toolflow.}

\begin{ParameterTable}
\Parameter{Hardware Platform}
{hw\_sys}
{Selects the board/chip to compile for.}

\Parameter{Include Linux add-on board support}
{ibob\_linux}
{Includes BORPH-capable Linux for IBOB.}

\Parameter{User IP Clock source}
{clk\_src}
{Selects the clock on which to run the System Generator circuit.}

\Parameter{GPIO Clock Pin I/O group}
{gpio\_clk\_io\_group}
{Selects GPIO type to use as clock input if using user clock on an IBOB.}

\Parameter{GPIO Clock Pin bit index}
{gpio\_clk\_bit\_index}
{Selects GPIO pin to use as clock input if using user clock on an IBOB.}

\Parameter{User IP Clock rate (MHz)}
{clk\_rate}
{Generates timing constraints for the design.}

\Parameter{Sample Period}
{sample\_period}
{Sample period for Simulink simulations.}

\Parameter{Synthesis Tool}
{synthesis\_tool}
{Selects the tool to use for synthesizing the design's netlist.}

\end{ParameterTable}

\begin{comment}
\begin{PortTable}
\Port{Port Name}{Port Direction}{Port data type}{Port Description}
\Port{Port Name}{in}{ufix\_x\_y}{Port Description}
\Port{Port Name}{in}{inherited}{Port Description}
\end{PortTable}
\end{comment}

\BlockDesc{The function of the XSG Core Config block is to set parameters for
the toolflow scripts. It supercedes the use of the Xilinx System Generator block
and has supplemental options for board-level parameters. Although a System
Generator block is still needed in all designs, the XSG Core Config block
automatically changes the System Generator block settings based on its own
parameters.

The settings in the XSG Core Config block are used to determine the system-%
level conditions of the SysGen design. It sets which of the toolflow-supported
boards the design is being compiled for, from which it determines what FPGA to
target, as well as clocking options like clock source and timing constraints.
The Sample Period and Synthesis Tool parameters are included in the block so
that all system-level options available in the System Generator block could
be handled by this single block.}
 
\end{document}
