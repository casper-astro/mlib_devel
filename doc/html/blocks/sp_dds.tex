\documentclass{article}
\oddsidemargin  0.0in
\evensidemargin 0.0in
\textwidth      6.5in
\usepackage{tabularx}
\usepackage{html}
\title{\textbf{CASPER Library} \\Reference Manual}
\newcommand{\Block}[6]{\section {#1 \emph{(#2)}} \label{#3} \textbf{Block Author}: #4 \\ \textbf{Document Author}: #5 \subsection*{Summary}#6}

\newenvironment{PortTable}{\subsection*{Ports}
\tabularx{6.5in}{|l|l|l|X|} \hline  \textbf{Port} & \textbf{Dir.} & \textbf{Data Type} & \textbf{Description} \\ \hline}{\endtabularx}

\newcommand{\Port}[4]{\emph{#1} & \lowercase{#2} & #3 & #4\\  \hline}

\newcommand{\BlockDesc}[1]{\subsection*{Description}#1}

\newenvironment{ParameterTable}{\subsection*{Mask Parameters}
\tabularx{6.5in}{|l|l|X|} \hline  \textbf{Parameter} & \textbf{Variable} & \textbf{Description} \\ \hline}{\endtabularx}

\newcommand{\Parameter}[3]{#1 & \emph{#2} & #3 \\ \hline}

\begin{htmlonly}
\newcommand{\tabularx}[3]{\begin{tabularx}{#1}{#2}{#3}}
\newcommand{\endtabularx}{\end{tabularx}}
\end{htmlonly}

\date{Last Updated \today}
\begin{document}
\maketitle

%\chapter{System Blocks}
%%%%Change Chapter%%%%%%%%
%\chapter{Signal Processing Blocks}

%\input{test.tex}
%\chapter{Communication Blocks}
%\end{document} 

\Block{DDS}{dds}{dds}{Aaron Parsons}{Ben Blackman}{Generates sines and cosines of different phases and outputs them in parallel.}


\begin{ParameterTable}
\Parameter{Frequency Divisions (M)}{freq\_div}{Denominator of the frequency.}
\Parameter{Frequency (? /M$*$2$*$pi)}{freq}{Numerator of the frequency.}
\Parameter{Parallel LOs}{num\_lo}{Number of parallel local oscillators.}
\Parameter{Bit Width}{n\_bits}{Bit width of the outputs.}
\Parameter{Latency}{latency}{Description}
\end{ParameterTable}

\begin{PortTable}
\Port{sinX}{OUT}{Fix\_(n\_bits)\_(n\_bits-1)}{Sine output corresponding to the Xth local oscillator.}
\Port{cosX}{OUT}{Fix\_(n\_bits)\_(n\_bits-1)}{Cosine output corresponding to the Xth local oscillator.}
\end{PortTable}

\BlockDesc{
\paragraph{Usage}
There are \textit{sin} and \textit{cos} outputs each equal to the minimum of \textit{num\_lo} and \textit{freq\_div}. If \textit{num\_lo} $>=$ \textit{freq\_div}/\textit{freq}, then the outputs will be \textit{lo\_const}s. Otherwise each output will oscillate depending on the values of \textit{freq\_div} and \textit{freq}. If the outputs oscillate, then there will be a latency of \textit{latency} and otherwise there will be zero latency.
 }
 
\end{document}
