\documentclass{article}
\oddsidemargin  0.0in
\evensidemargin 0.0in
\textwidth      6.5in
\usepackage{tabularx}
\usepackage{html}
\title{\textbf{CASPER Library} \\Reference Manual}
\newcommand{\Block}[6]{\section {#1 \emph{(#2)}} \label{#3} \textbf{Block Author}: #4 \\ \textbf{Document Author}: #5 \subsection*{Summary}#6}

\newenvironment{PortTable}{\subsection*{Ports}
\tabularx{6.5in}{|l|l|l|X|} \hline  \textbf{Port} & \textbf{Dir.} & \textbf{Data Type} & \textbf{Description} \\ \hline}{\endtabularx}

\newcommand{\Port}[4]{\emph{#1} & \lowercase{#2} & #3 & #4\\  \hline}

\newcommand{\BlockDesc}[1]{\subsection*{Description}#1}

\newenvironment{ParameterTable}{\subsection*{Mask Parameters}
\tabularx{6.5in}{|l|l|X|} \hline  \textbf{Parameter} & \textbf{Variable} & \textbf{Description} \\ \hline}{\endtabularx}

\newcommand{\Parameter}[3]{#1 & \emph{#2} & #3 \\ \hline}

\begin{htmlonly}
\newcommand{\tabularx}[3]{\begin{tabularx}{#1}{#2}{#3}}
\newcommand{\endtabularx}{\end{tabularx}}
\end{htmlonly}

\date{Last Updated \today}
\begin{document}
\maketitle

%\chapter{System Blocks}
%%%%Change Chapter%%%%%%%%
%\chapter{Signal Processing Blocks}

%\input{test.tex}
%\chapter{Communication Blocks}
%\end{document} 
\Block{Square Transposer}
{square\_transposer}
{squaretransposer}
{Aaron Parsons}
{Aaron Parsons}
{Presents a number of parallel inputs serially on the same number of output lines.}

\begin{ParameterTable}
\Parameter{Number of inputs}{n\_inputs}{The number of parallel inputs (and outputs).}
\end{ParameterTable}

\begin{PortTable}
\Port{sync}{in}{Boolean}{Indicates the next clock cycle contains valid data}
\Port{In}{in}{Inherited}{The stream(s) to be transposed.}
\Port{sync\_out}{out}{Boolean}{Indicates that data out will be valid next clock cycle.}
\Port{Out}{out}{Inherited}{The transposed stream(s).}
\end{PortTable}

\BlockDesc{Presents a number of parallel inputs serially on the same number of output lines.  After a sync pulse, all of the parallel streams input to the square transposer will appear serially on Out1.  The all parallel data from the following clock cycle will appear serially on Out2, and so on.  All the data output
(Out1, Out2, etc.) appear aligned:

\begin{table*}[h]
	\centering
		\begin{tabular}{|r|rrrr|r|rrrr|r|}
		\hline
In1 & d12 & d8  & d4 & d0 & $\rightarrow$ & d3  & d2  & d1  & d0  & Out1 \\ 
In2 & d13 & d9  & d5 & d1 & $\rightarrow$ & d7  & d6  & d5  & d4  & Out2 \\ 
In3 & d14 & d10 & d6 & d2 & $\rightarrow$ & d11 & d10 & d9  & d8  & Out3 \\ 
In4 & d15 & d11 & d7 & d3 & $\rightarrow$ & d15 & d14 & d13 & d12 & Out4 \\ \hline
		\end{tabular}
\end{table*}


%\begin{matrix}[r|rrrr|r|rrrr|r] \hline
%In1 & d12 & d8  & d4 & d0 & \rightarrow & d3  & d2  & d1  & d0  & Out1 \hline
%In2 & d13 & d9  & d5 & d1 & \rightarrow & d7  & d6  & d5  & d4  & Out2 \hline
%In3 & d14 & d10 & d6 & d2 & \rightarrow & d11 & d10 & d9  & d8  & Out3 \hline
%In4 & d15 & d11 & d7 & d3 & \rightarrow & d15 & d14 & d13 & d12 & Out4 \hline
%\end{matrix}}

} 
\end{document}
